
% Default to the notebook output style

    


% Inherit from the specified cell style.




    
\documentclass[11pt]{article}

    
    
    \usepackage[T1]{fontenc}
    % Nicer default font (+ math font) than Computer Modern for most use cases
    \usepackage{mathpazo}

    % Basic figure setup, for now with no caption control since it's done
    % automatically by Pandoc (which extracts ![](path) syntax from Markdown).
    \usepackage{graphicx}
    % We will generate all images so they have a width \maxwidth. This means
    % that they will get their normal width if they fit onto the page, but
    % are scaled down if they would overflow the margins.
    \makeatletter
    \def\maxwidth{\ifdim\Gin@nat@width>\linewidth\linewidth
    \else\Gin@nat@width\fi}
    \makeatother
    \let\Oldincludegraphics\includegraphics
    % Set max figure width to be 80% of text width, for now hardcoded.
    \renewcommand{\includegraphics}[1]{\Oldincludegraphics[width=.8\maxwidth]{#1}}
    % Ensure that by default, figures have no caption (until we provide a
    % proper Figure object with a Caption API and a way to capture that
    % in the conversion process - todo).
    \usepackage{caption}
    \DeclareCaptionLabelFormat{nolabel}{}
    \captionsetup{labelformat=nolabel}

    \usepackage{adjustbox} % Used to constrain images to a maximum size 
    \usepackage{xcolor} % Allow colors to be defined
    \usepackage{enumerate} % Needed for markdown enumerations to work
    \usepackage{geometry} % Used to adjust the document margins
    \usepackage{amsmath} % Equations
    \usepackage{amssymb} % Equations
    \usepackage{textcomp} % defines textquotesingle
    % Hack from http://tex.stackexchange.com/a/47451/13684:
    \AtBeginDocument{%
        \def\PYZsq{\textquotesingle}% Upright quotes in Pygmentized code
    }
    \usepackage{upquote} % Upright quotes for verbatim code
    \usepackage{eurosym} % defines \euro
    \usepackage[mathletters]{ucs} % Extended unicode (utf-8) support
    \usepackage[utf8x]{inputenc} % Allow utf-8 characters in the tex document
    \usepackage{fancyvrb} % verbatim replacement that allows latex
    \usepackage{grffile} % extends the file name processing of package graphics 
                         % to support a larger range 
    % The hyperref package gives us a pdf with properly built
    % internal navigation ('pdf bookmarks' for the table of contents,
    % internal cross-reference links, web links for URLs, etc.)
    \usepackage{hyperref}
    \usepackage{longtable} % longtable support required by pandoc >1.10
    \usepackage{booktabs}  % table support for pandoc > 1.12.2
    \usepackage[inline]{enumitem} % IRkernel/repr support (it uses the enumerate* environment)
    \usepackage[normalem]{ulem} % ulem is needed to support strikethroughs (\sout)
                                % normalem makes italics be italics, not underlines
    

    
    
    % Colors for the hyperref package
    \definecolor{urlcolor}{rgb}{0,.145,.698}
    \definecolor{linkcolor}{rgb}{.71,0.21,0.01}
    \definecolor{citecolor}{rgb}{.12,.54,.11}

    % ANSI colors
    \definecolor{ansi-black}{HTML}{3E424D}
    \definecolor{ansi-black-intense}{HTML}{282C36}
    \definecolor{ansi-red}{HTML}{E75C58}
    \definecolor{ansi-red-intense}{HTML}{B22B31}
    \definecolor{ansi-green}{HTML}{00A250}
    \definecolor{ansi-green-intense}{HTML}{007427}
    \definecolor{ansi-yellow}{HTML}{DDB62B}
    \definecolor{ansi-yellow-intense}{HTML}{B27D12}
    \definecolor{ansi-blue}{HTML}{208FFB}
    \definecolor{ansi-blue-intense}{HTML}{0065CA}
    \definecolor{ansi-magenta}{HTML}{D160C4}
    \definecolor{ansi-magenta-intense}{HTML}{A03196}
    \definecolor{ansi-cyan}{HTML}{60C6C8}
    \definecolor{ansi-cyan-intense}{HTML}{258F8F}
    \definecolor{ansi-white}{HTML}{C5C1B4}
    \definecolor{ansi-white-intense}{HTML}{A1A6B2}

    % commands and environments needed by pandoc snippets
    % extracted from the output of `pandoc -s`
    \providecommand{\tightlist}{%
      \setlength{\itemsep}{0pt}\setlength{\parskip}{0pt}}
    \DefineVerbatimEnvironment{Highlighting}{Verbatim}{commandchars=\\\{\}}
    % Add ',fontsize=\small' for more characters per line
    \newenvironment{Shaded}{}{}
    \newcommand{\KeywordTok}[1]{\textcolor[rgb]{0.00,0.44,0.13}{\textbf{{#1}}}}
    \newcommand{\DataTypeTok}[1]{\textcolor[rgb]{0.56,0.13,0.00}{{#1}}}
    \newcommand{\DecValTok}[1]{\textcolor[rgb]{0.25,0.63,0.44}{{#1}}}
    \newcommand{\BaseNTok}[1]{\textcolor[rgb]{0.25,0.63,0.44}{{#1}}}
    \newcommand{\FloatTok}[1]{\textcolor[rgb]{0.25,0.63,0.44}{{#1}}}
    \newcommand{\CharTok}[1]{\textcolor[rgb]{0.25,0.44,0.63}{{#1}}}
    \newcommand{\StringTok}[1]{\textcolor[rgb]{0.25,0.44,0.63}{{#1}}}
    \newcommand{\CommentTok}[1]{\textcolor[rgb]{0.38,0.63,0.69}{\textit{{#1}}}}
    \newcommand{\OtherTok}[1]{\textcolor[rgb]{0.00,0.44,0.13}{{#1}}}
    \newcommand{\AlertTok}[1]{\textcolor[rgb]{1.00,0.00,0.00}{\textbf{{#1}}}}
    \newcommand{\FunctionTok}[1]{\textcolor[rgb]{0.02,0.16,0.49}{{#1}}}
    \newcommand{\RegionMarkerTok}[1]{{#1}}
    \newcommand{\ErrorTok}[1]{\textcolor[rgb]{1.00,0.00,0.00}{\textbf{{#1}}}}
    \newcommand{\NormalTok}[1]{{#1}}
    
    % Additional commands for more recent versions of Pandoc
    \newcommand{\ConstantTok}[1]{\textcolor[rgb]{0.53,0.00,0.00}{{#1}}}
    \newcommand{\SpecialCharTok}[1]{\textcolor[rgb]{0.25,0.44,0.63}{{#1}}}
    \newcommand{\VerbatimStringTok}[1]{\textcolor[rgb]{0.25,0.44,0.63}{{#1}}}
    \newcommand{\SpecialStringTok}[1]{\textcolor[rgb]{0.73,0.40,0.53}{{#1}}}
    \newcommand{\ImportTok}[1]{{#1}}
    \newcommand{\DocumentationTok}[1]{\textcolor[rgb]{0.73,0.13,0.13}{\textit{{#1}}}}
    \newcommand{\AnnotationTok}[1]{\textcolor[rgb]{0.38,0.63,0.69}{\textbf{\textit{{#1}}}}}
    \newcommand{\CommentVarTok}[1]{\textcolor[rgb]{0.38,0.63,0.69}{\textbf{\textit{{#1}}}}}
    \newcommand{\VariableTok}[1]{\textcolor[rgb]{0.10,0.09,0.49}{{#1}}}
    \newcommand{\ControlFlowTok}[1]{\textcolor[rgb]{0.00,0.44,0.13}{\textbf{{#1}}}}
    \newcommand{\OperatorTok}[1]{\textcolor[rgb]{0.40,0.40,0.40}{{#1}}}
    \newcommand{\BuiltInTok}[1]{{#1}}
    \newcommand{\ExtensionTok}[1]{{#1}}
    \newcommand{\PreprocessorTok}[1]{\textcolor[rgb]{0.74,0.48,0.00}{{#1}}}
    \newcommand{\AttributeTok}[1]{\textcolor[rgb]{0.49,0.56,0.16}{{#1}}}
    \newcommand{\InformationTok}[1]{\textcolor[rgb]{0.38,0.63,0.69}{\textbf{\textit{{#1}}}}}
    \newcommand{\WarningTok}[1]{\textcolor[rgb]{0.38,0.63,0.69}{\textbf{\textit{{#1}}}}}
    
    
    % Define a nice break command that doesn't care if a line doesn't already
    % exist.
    \def\br{\hspace*{\fill} \\* }
    % Math Jax compatability definitions
    \def\gt{>}
    \def\lt{<}
    % Document parameters
    \title{Fisher}
    
    
    

    % Pygments definitions
    
\makeatletter
\def\PY@reset{\let\PY@it=\relax \let\PY@bf=\relax%
    \let\PY@ul=\relax \let\PY@tc=\relax%
    \let\PY@bc=\relax \let\PY@ff=\relax}
\def\PY@tok#1{\csname PY@tok@#1\endcsname}
\def\PY@toks#1+{\ifx\relax#1\empty\else%
    \PY@tok{#1}\expandafter\PY@toks\fi}
\def\PY@do#1{\PY@bc{\PY@tc{\PY@ul{%
    \PY@it{\PY@bf{\PY@ff{#1}}}}}}}
\def\PY#1#2{\PY@reset\PY@toks#1+\relax+\PY@do{#2}}

\expandafter\def\csname PY@tok@w\endcsname{\def\PY@tc##1{\textcolor[rgb]{0.73,0.73,0.73}{##1}}}
\expandafter\def\csname PY@tok@c\endcsname{\let\PY@it=\textit\def\PY@tc##1{\textcolor[rgb]{0.25,0.50,0.50}{##1}}}
\expandafter\def\csname PY@tok@cp\endcsname{\def\PY@tc##1{\textcolor[rgb]{0.74,0.48,0.00}{##1}}}
\expandafter\def\csname PY@tok@k\endcsname{\let\PY@bf=\textbf\def\PY@tc##1{\textcolor[rgb]{0.00,0.50,0.00}{##1}}}
\expandafter\def\csname PY@tok@kp\endcsname{\def\PY@tc##1{\textcolor[rgb]{0.00,0.50,0.00}{##1}}}
\expandafter\def\csname PY@tok@kt\endcsname{\def\PY@tc##1{\textcolor[rgb]{0.69,0.00,0.25}{##1}}}
\expandafter\def\csname PY@tok@o\endcsname{\def\PY@tc##1{\textcolor[rgb]{0.40,0.40,0.40}{##1}}}
\expandafter\def\csname PY@tok@ow\endcsname{\let\PY@bf=\textbf\def\PY@tc##1{\textcolor[rgb]{0.67,0.13,1.00}{##1}}}
\expandafter\def\csname PY@tok@nb\endcsname{\def\PY@tc##1{\textcolor[rgb]{0.00,0.50,0.00}{##1}}}
\expandafter\def\csname PY@tok@nf\endcsname{\def\PY@tc##1{\textcolor[rgb]{0.00,0.00,1.00}{##1}}}
\expandafter\def\csname PY@tok@nc\endcsname{\let\PY@bf=\textbf\def\PY@tc##1{\textcolor[rgb]{0.00,0.00,1.00}{##1}}}
\expandafter\def\csname PY@tok@nn\endcsname{\let\PY@bf=\textbf\def\PY@tc##1{\textcolor[rgb]{0.00,0.00,1.00}{##1}}}
\expandafter\def\csname PY@tok@ne\endcsname{\let\PY@bf=\textbf\def\PY@tc##1{\textcolor[rgb]{0.82,0.25,0.23}{##1}}}
\expandafter\def\csname PY@tok@nv\endcsname{\def\PY@tc##1{\textcolor[rgb]{0.10,0.09,0.49}{##1}}}
\expandafter\def\csname PY@tok@no\endcsname{\def\PY@tc##1{\textcolor[rgb]{0.53,0.00,0.00}{##1}}}
\expandafter\def\csname PY@tok@nl\endcsname{\def\PY@tc##1{\textcolor[rgb]{0.63,0.63,0.00}{##1}}}
\expandafter\def\csname PY@tok@ni\endcsname{\let\PY@bf=\textbf\def\PY@tc##1{\textcolor[rgb]{0.60,0.60,0.60}{##1}}}
\expandafter\def\csname PY@tok@na\endcsname{\def\PY@tc##1{\textcolor[rgb]{0.49,0.56,0.16}{##1}}}
\expandafter\def\csname PY@tok@nt\endcsname{\let\PY@bf=\textbf\def\PY@tc##1{\textcolor[rgb]{0.00,0.50,0.00}{##1}}}
\expandafter\def\csname PY@tok@nd\endcsname{\def\PY@tc##1{\textcolor[rgb]{0.67,0.13,1.00}{##1}}}
\expandafter\def\csname PY@tok@s\endcsname{\def\PY@tc##1{\textcolor[rgb]{0.73,0.13,0.13}{##1}}}
\expandafter\def\csname PY@tok@sd\endcsname{\let\PY@it=\textit\def\PY@tc##1{\textcolor[rgb]{0.73,0.13,0.13}{##1}}}
\expandafter\def\csname PY@tok@si\endcsname{\let\PY@bf=\textbf\def\PY@tc##1{\textcolor[rgb]{0.73,0.40,0.53}{##1}}}
\expandafter\def\csname PY@tok@se\endcsname{\let\PY@bf=\textbf\def\PY@tc##1{\textcolor[rgb]{0.73,0.40,0.13}{##1}}}
\expandafter\def\csname PY@tok@sr\endcsname{\def\PY@tc##1{\textcolor[rgb]{0.73,0.40,0.53}{##1}}}
\expandafter\def\csname PY@tok@ss\endcsname{\def\PY@tc##1{\textcolor[rgb]{0.10,0.09,0.49}{##1}}}
\expandafter\def\csname PY@tok@sx\endcsname{\def\PY@tc##1{\textcolor[rgb]{0.00,0.50,0.00}{##1}}}
\expandafter\def\csname PY@tok@m\endcsname{\def\PY@tc##1{\textcolor[rgb]{0.40,0.40,0.40}{##1}}}
\expandafter\def\csname PY@tok@gh\endcsname{\let\PY@bf=\textbf\def\PY@tc##1{\textcolor[rgb]{0.00,0.00,0.50}{##1}}}
\expandafter\def\csname PY@tok@gu\endcsname{\let\PY@bf=\textbf\def\PY@tc##1{\textcolor[rgb]{0.50,0.00,0.50}{##1}}}
\expandafter\def\csname PY@tok@gd\endcsname{\def\PY@tc##1{\textcolor[rgb]{0.63,0.00,0.00}{##1}}}
\expandafter\def\csname PY@tok@gi\endcsname{\def\PY@tc##1{\textcolor[rgb]{0.00,0.63,0.00}{##1}}}
\expandafter\def\csname PY@tok@gr\endcsname{\def\PY@tc##1{\textcolor[rgb]{1.00,0.00,0.00}{##1}}}
\expandafter\def\csname PY@tok@ge\endcsname{\let\PY@it=\textit}
\expandafter\def\csname PY@tok@gs\endcsname{\let\PY@bf=\textbf}
\expandafter\def\csname PY@tok@gp\endcsname{\let\PY@bf=\textbf\def\PY@tc##1{\textcolor[rgb]{0.00,0.00,0.50}{##1}}}
\expandafter\def\csname PY@tok@go\endcsname{\def\PY@tc##1{\textcolor[rgb]{0.53,0.53,0.53}{##1}}}
\expandafter\def\csname PY@tok@gt\endcsname{\def\PY@tc##1{\textcolor[rgb]{0.00,0.27,0.87}{##1}}}
\expandafter\def\csname PY@tok@err\endcsname{\def\PY@bc##1{\setlength{\fboxsep}{0pt}\fcolorbox[rgb]{1.00,0.00,0.00}{1,1,1}{\strut ##1}}}
\expandafter\def\csname PY@tok@kc\endcsname{\let\PY@bf=\textbf\def\PY@tc##1{\textcolor[rgb]{0.00,0.50,0.00}{##1}}}
\expandafter\def\csname PY@tok@kd\endcsname{\let\PY@bf=\textbf\def\PY@tc##1{\textcolor[rgb]{0.00,0.50,0.00}{##1}}}
\expandafter\def\csname PY@tok@kn\endcsname{\let\PY@bf=\textbf\def\PY@tc##1{\textcolor[rgb]{0.00,0.50,0.00}{##1}}}
\expandafter\def\csname PY@tok@kr\endcsname{\let\PY@bf=\textbf\def\PY@tc##1{\textcolor[rgb]{0.00,0.50,0.00}{##1}}}
\expandafter\def\csname PY@tok@bp\endcsname{\def\PY@tc##1{\textcolor[rgb]{0.00,0.50,0.00}{##1}}}
\expandafter\def\csname PY@tok@fm\endcsname{\def\PY@tc##1{\textcolor[rgb]{0.00,0.00,1.00}{##1}}}
\expandafter\def\csname PY@tok@vc\endcsname{\def\PY@tc##1{\textcolor[rgb]{0.10,0.09,0.49}{##1}}}
\expandafter\def\csname PY@tok@vg\endcsname{\def\PY@tc##1{\textcolor[rgb]{0.10,0.09,0.49}{##1}}}
\expandafter\def\csname PY@tok@vi\endcsname{\def\PY@tc##1{\textcolor[rgb]{0.10,0.09,0.49}{##1}}}
\expandafter\def\csname PY@tok@vm\endcsname{\def\PY@tc##1{\textcolor[rgb]{0.10,0.09,0.49}{##1}}}
\expandafter\def\csname PY@tok@sa\endcsname{\def\PY@tc##1{\textcolor[rgb]{0.73,0.13,0.13}{##1}}}
\expandafter\def\csname PY@tok@sb\endcsname{\def\PY@tc##1{\textcolor[rgb]{0.73,0.13,0.13}{##1}}}
\expandafter\def\csname PY@tok@sc\endcsname{\def\PY@tc##1{\textcolor[rgb]{0.73,0.13,0.13}{##1}}}
\expandafter\def\csname PY@tok@dl\endcsname{\def\PY@tc##1{\textcolor[rgb]{0.73,0.13,0.13}{##1}}}
\expandafter\def\csname PY@tok@s2\endcsname{\def\PY@tc##1{\textcolor[rgb]{0.73,0.13,0.13}{##1}}}
\expandafter\def\csname PY@tok@sh\endcsname{\def\PY@tc##1{\textcolor[rgb]{0.73,0.13,0.13}{##1}}}
\expandafter\def\csname PY@tok@s1\endcsname{\def\PY@tc##1{\textcolor[rgb]{0.73,0.13,0.13}{##1}}}
\expandafter\def\csname PY@tok@mb\endcsname{\def\PY@tc##1{\textcolor[rgb]{0.40,0.40,0.40}{##1}}}
\expandafter\def\csname PY@tok@mf\endcsname{\def\PY@tc##1{\textcolor[rgb]{0.40,0.40,0.40}{##1}}}
\expandafter\def\csname PY@tok@mh\endcsname{\def\PY@tc##1{\textcolor[rgb]{0.40,0.40,0.40}{##1}}}
\expandafter\def\csname PY@tok@mi\endcsname{\def\PY@tc##1{\textcolor[rgb]{0.40,0.40,0.40}{##1}}}
\expandafter\def\csname PY@tok@il\endcsname{\def\PY@tc##1{\textcolor[rgb]{0.40,0.40,0.40}{##1}}}
\expandafter\def\csname PY@tok@mo\endcsname{\def\PY@tc##1{\textcolor[rgb]{0.40,0.40,0.40}{##1}}}
\expandafter\def\csname PY@tok@ch\endcsname{\let\PY@it=\textit\def\PY@tc##1{\textcolor[rgb]{0.25,0.50,0.50}{##1}}}
\expandafter\def\csname PY@tok@cm\endcsname{\let\PY@it=\textit\def\PY@tc##1{\textcolor[rgb]{0.25,0.50,0.50}{##1}}}
\expandafter\def\csname PY@tok@cpf\endcsname{\let\PY@it=\textit\def\PY@tc##1{\textcolor[rgb]{0.25,0.50,0.50}{##1}}}
\expandafter\def\csname PY@tok@c1\endcsname{\let\PY@it=\textit\def\PY@tc##1{\textcolor[rgb]{0.25,0.50,0.50}{##1}}}
\expandafter\def\csname PY@tok@cs\endcsname{\let\PY@it=\textit\def\PY@tc##1{\textcolor[rgb]{0.25,0.50,0.50}{##1}}}

\def\PYZbs{\char`\\}
\def\PYZus{\char`\_}
\def\PYZob{\char`\{}
\def\PYZcb{\char`\}}
\def\PYZca{\char`\^}
\def\PYZam{\char`\&}
\def\PYZlt{\char`\<}
\def\PYZgt{\char`\>}
\def\PYZsh{\char`\#}
\def\PYZpc{\char`\%}
\def\PYZdl{\char`\$}
\def\PYZhy{\char`\-}
\def\PYZsq{\char`\'}
\def\PYZdq{\char`\"}
\def\PYZti{\char`\~}
% for compatibility with earlier versions
\def\PYZat{@}
\def\PYZlb{[}
\def\PYZrb{]}
\makeatother


    % Exact colors from NB
    \definecolor{incolor}{rgb}{0.0, 0.0, 0.5}
    \definecolor{outcolor}{rgb}{0.545, 0.0, 0.0}



    
    % Prevent overflowing lines due to hard-to-break entities
    \sloppy 
    % Setup hyperref package
    \hypersetup{
      breaklinks=true,  % so long urls are correctly broken across lines
      colorlinks=true,
      urlcolor=urlcolor,
      linkcolor=linkcolor,
      citecolor=citecolor,
      }
    % Slightly bigger margins than the latex defaults
    
    \geometry{verbose,tmargin=1in,bmargin=1in,lmargin=1in,rmargin=1in}
    
    

    \begin{document}
    
    
    \maketitle
    
    

    
    \section{Mustererkennung Übung 4}\label{mustererkennung-uxfcbung-4}

Samuel Gfrörer Marcel Schmidt 

    \begin{Verbatim}[commandchars=\\\{\}]
{\color{incolor}In [{\color{incolor}50}]:} \PY{k+kn}{import} \PY{n+nn}{numpy} \PY{k}{as} \PY{n+nn}{np}
         \PY{k+kn}{import} \PY{n+nn}{random} \PY{k}{as} \PY{n+nn}{rn}
         \PY{k+kn}{import} \PY{n+nn}{pylab}
         \PY{k+kn}{from} \PY{n+nn}{numpy}\PY{n+nn}{.}\PY{n+nn}{linalg} \PY{k}{import} \PY{n}{pinv}
         \PY{k+kn}{from} \PY{n+nn}{numpy}\PY{n+nn}{.}\PY{n+nn}{linalg} \PY{k}{import} \PY{n}{norm}
         \PY{k+kn}{from} \PY{n+nn}{sklearn}\PY{n+nn}{.}\PY{n+nn}{cross\PYZus{}validation} \PY{k}{import} \PY{n}{train\PYZus{}test\PYZus{}split}
         
         \PY{k}{def} \PY{n+nf}{getData}\PY{p}{(}\PY{p}{)}\PY{p}{:}
             \PY{n}{file} \PY{o}{=} \PY{n+nb}{open}\PY{p}{(}\PY{l+s+s2}{\PYZdq{}}\PY{l+s+s2}{res/spambase.data}\PY{l+s+s2}{\PYZdq{}}\PY{p}{,} \PY{l+s+s2}{\PYZdq{}}\PY{l+s+s2}{r}\PY{l+s+s2}{\PYZdq{}}\PY{p}{)} 
             \PY{n}{Data}\PY{o}{=}\PY{p}{[}\PY{p}{]}    
             \PY{k}{for} \PY{n}{line} \PY{o+ow}{in} \PY{n}{file}\PY{p}{:}
                 \PY{n}{floatLine} \PY{o}{=} \PY{p}{[}\PY{p}{]}
                 \PY{n}{tmpLine} \PY{o}{=} \PY{n}{line}\PY{o}{.}\PY{n}{split}\PY{p}{(}\PY{l+s+s1}{\PYZsq{}}\PY{l+s+s1}{,}\PY{l+s+s1}{\PYZsq{}}\PY{p}{)}
                 \PY{n}{Linelen} \PY{o}{=} \PY{n+nb}{len}\PY{p}{(}\PY{n}{tmpLine}\PY{p}{)}            
                 \PY{k}{for} \PY{n}{i} \PY{o+ow}{in} \PY{n+nb}{range}\PY{p}{(}\PY{l+m+mi}{0}\PY{p}{,}\PY{n}{Linelen}\PY{p}{)}\PY{p}{:}
                     \PY{n}{floatLine}\PY{o}{.}\PY{n}{append}\PY{p}{(}\PY{n+nb}{float}\PY{p}{(}\PY{n}{tmpLine}\PY{p}{[}\PY{n}{i}\PY{p}{]}\PY{p}{)}\PY{p}{)}            
                 \PY{n}{Data}\PY{o}{.}\PY{n}{append}\PY{p}{(}\PY{n}{floatLine}\PY{p}{)}                
             \PY{n}{file}\PY{o}{.}\PY{n}{closed}   
             \PY{k}{return} \PY{n}{Data}
         
         \PY{k}{def} \PY{n+nf}{getVectors}\PY{p}{(}\PY{n}{label}\PY{p}{,} \PY{n}{data}\PY{p}{)}\PY{p}{:}
             \PY{n}{vectors} \PY{o}{=} \PY{p}{[}\PY{p}{]}
             \PY{k}{for} \PY{n}{line} \PY{o+ow}{in} \PY{n}{data}\PY{p}{:}
                 \PY{c+c1}{\PYZsh{}print(line[\PYZhy{}1])}
                 \PY{k}{if}\PY{p}{(}\PY{n}{line}\PY{p}{[}\PY{o}{\PYZhy{}}\PY{l+m+mi}{1}\PY{p}{]} \PY{o}{==} \PY{n+nb}{float}\PY{p}{(}\PY{n}{label}\PY{p}{)}\PY{p}{)}\PY{p}{:}
                     \PY{n}{vectors}\PY{o}{.}\PY{n}{append}\PY{p}{(}\PY{n}{line}\PY{p}{[}\PY{p}{:}\PY{o}{\PYZhy{}}\PY{l+m+mi}{1}\PY{p}{]}\PY{p}{)}
             \PY{k}{return} \PY{n}{np}\PY{o}{.}\PY{n}{array}\PY{p}{(}\PY{n}{vectors}\PY{p}{)}
\end{Verbatim}


    \subsection{Hilfsfunktionen}\label{hilfsfunktionen}

\begin{itemize}
\tightlist
\item
  MittelPunktVector(VecList): Berechnung des Mittelpunktvektors aus
  einer Liste von Vektoren \(\vec v_0\) bis \(\vec v_N\)
\end{itemize}

\[\vec M = \frac{1}{N} \sum_{i = 0}^N \vec v_i \]

\begin{itemize}
\tightlist
\item
  getKovarianzMatrix(Data, M): Berechnung der Kovarianzmatrix \(\Sigma\)
  aus einer Liste von Vektoren \(\vec v_0\) bis \(\vec v_N\) und
  zugehörigem Mittelpunktvektor \(\vec M\)
\end{itemize}

\[\Sigma = \frac{1}{N}\sum_{i = 0}^N (\vec v_i - \vec M)(\vec v_i - \vec M)^T \]

\begin{itemize}
\tightlist
\item
  gauß(x, mean, var): Berechnung der Wahrscheinlichkeitsdichte p(x) für
  Mittelwert \(\mu\) und Varianz \(\sigma^2\) mit gaußscher
  Normalverteilung
\end{itemize}

\[p(x) = \frac{1}{ \sqrt{2  \pi \sigma^2} } e^{-\frac{(x - \mu)^2}{2 \sigma^2}} \]

    \begin{Verbatim}[commandchars=\\\{\}]
{\color{incolor}In [{\color{incolor}51}]:} \PY{k}{def} \PY{n+nf}{MittelPunktVector}\PY{p}{(}\PY{n}{VecList}\PY{p}{)}\PY{p}{:}    
             \PY{n}{M} \PY{o}{=} \PY{n}{np}\PY{o}{.}\PY{n}{array}\PY{p}{(}\PY{n}{VecList}\PY{p}{[}\PY{l+m+mi}{0}\PY{p}{]}\PY{p}{)}
             \PY{n}{n} \PY{o}{=} \PY{n+nb}{len}\PY{p}{(}\PY{n}{VecList}\PY{p}{)}    
             \PY{k}{for} \PY{n}{i} \PY{o+ow}{in} \PY{n+nb}{range}\PY{p}{(}\PY{l+m+mi}{1}\PY{p}{,}\PY{n}{n}\PY{p}{)}\PY{p}{:}
                 \PY{n}{tmpVec} \PY{o}{=} \PY{n}{np}\PY{o}{.}\PY{n}{array}\PY{p}{(}\PY{n}{VecList}\PY{p}{[}\PY{n}{i}\PY{p}{]}\PY{p}{)}
                 \PY{n}{M} \PY{o}{=} \PY{n}{M} \PY{o}{+} \PY{n}{tmpVec}
             \PY{n}{M} \PY{o}{=} \PY{n}{M} \PY{o}{/} \PY{p}{(}\PY{n}{n}\PY{p}{)} 
             \PY{k}{return} \PY{n}{M}
         
         \PY{k}{def} \PY{n+nf}{getKovarianzMatrix}\PY{p}{(}\PY{n}{Data}\PY{p}{,} \PY{n}{M}\PY{p}{)}\PY{p}{:}  
             \PY{n}{n} \PY{o}{=} \PY{n+nb}{len}\PY{p}{(}\PY{n}{Data}\PY{p}{)}    
             \PY{n}{x} \PY{o}{=} \PY{n}{np}\PY{o}{.}\PY{n}{matrix}\PY{p}{(}\PY{n}{Data}\PY{p}{[}\PY{l+m+mi}{0}\PY{p}{]}\PY{p}{)}
             \PY{n}{X} \PY{o}{=} \PY{n}{np}\PY{o}{.}\PY{n}{transpose}\PY{p}{(}\PY{p}{(}\PY{n}{x} \PY{o}{\PYZhy{}} \PY{n}{M}\PY{p}{)}\PY{p}{)} \PY{o}{@} \PY{p}{(}\PY{n}{x} \PY{o}{\PYZhy{}} \PY{n}{M}\PY{p}{)}   
             \PY{k}{for} \PY{n}{i} \PY{o+ow}{in} \PY{n+nb}{range}\PY{p}{(}\PY{l+m+mi}{1}\PY{p}{,}\PY{n}{n}\PY{p}{)}\PY{p}{:}        
                 \PY{n}{x} \PY{o}{=} \PY{n}{np}\PY{o}{.}\PY{n}{matrix}\PY{p}{(}\PY{n}{Data}\PY{p}{[}\PY{n}{i}\PY{p}{]}\PY{p}{)}
                 \PY{n}{tmpX} \PY{o}{=} \PY{n}{np}\PY{o}{.}\PY{n}{transpose}\PY{p}{(}\PY{p}{(}\PY{n}{x} \PY{o}{\PYZhy{}} \PY{n}{M}\PY{p}{)}\PY{p}{)} \PY{o}{@} \PY{p}{(}\PY{n}{x} \PY{o}{\PYZhy{}} \PY{n}{M}\PY{p}{)}        
                 \PY{n}{X} \PY{o}{=} \PY{n}{X} \PY{o}{+} \PY{n}{tmpX}        
         
             \PY{k}{return} \PY{p}{(}\PY{n}{X} \PY{o}{/} \PY{n}{n}\PY{p}{)}
         
         \PY{k}{def} \PY{n+nf}{gauß}\PY{p}{(}\PY{n}{x}\PY{p}{,} \PY{n}{mean}\PY{p}{,} \PY{n}{var}\PY{p}{)}\PY{p}{:}
             \PY{k}{return} \PY{p}{(}\PY{l+m+mi}{1} \PY{o}{/} \PY{n}{np}\PY{o}{.}\PY{n}{sqrt}\PY{p}{(}\PY{l+m+mi}{2} \PY{o}{*} \PY{n}{np}\PY{o}{.}\PY{n}{pi} \PY{o}{*} \PY{n}{var} \PY{p}{)}\PY{p}{)} \PY{o}{*} \PY{n}{np}\PY{o}{.}\PY{n}{exp}\PY{p}{(} \PY{o}{\PYZhy{}}\PY{p}{(}\PY{p}{(}\PY{n}{x} \PY{o}{\PYZhy{}} \PY{n}{mean}\PY{p}{)}\PY{o}{*}\PY{o}{*}\PY{l+m+mi}{2}\PY{o}{/}\PY{p}{(}\PY{l+m+mi}{2} \PY{o}{*} \PY{n}{var}\PY{p}{)}\PY{p}{)}\PY{p}{)}
\end{Verbatim}


    \subsection{Berechnung von Projektionsvektor alpha, Mittelwert und
Varianz}\label{berechnung-von-projektionsvektor-alpha-mittelwert-und-varianz}

    \begin{itemize}
\tightlist
\item
  Einlesen der Vektoren
\item
  Aufteilen in positive und negative Klassen
\item
  Zufälliges Aufspalten in 20\% Testdaten und 80\% Trainingsdaten
\end{itemize}

    \begin{Verbatim}[commandchars=\\\{\}]
{\color{incolor}In [{\color{incolor}52}]:} \PY{n}{data} \PY{o}{=} \PY{n}{getData}\PY{p}{(}\PY{p}{)}
         
         \PY{n}{TrainVecsP} \PY{o}{=} \PY{n}{getVectors}\PY{p}{(}\PY{l+m+mi}{1}\PY{p}{,} \PY{n}{data}\PY{p}{)}
         \PY{n}{TrainVecsN} \PY{o}{=} \PY{n}{getVectors}\PY{p}{(}\PY{l+m+mi}{0}\PY{p}{,} \PY{n}{data}\PY{p}{)}
         
         \PY{n}{TrainVecsP} \PY{p}{,}\PY{n}{TestVecsP} \PY{o}{=} \PY{n}{train\PYZus{}test\PYZus{}split}\PY{p}{(}\PY{n}{TrainVecsP}\PY{p}{,}\PY{n}{test\PYZus{}size}\PY{o}{=}\PY{l+m+mf}{0.2}\PY{p}{)}
         \PY{n}{TrainVecsN} \PY{p}{,}\PY{n}{TestVecsN} \PY{o}{=} \PY{n}{train\PYZus{}test\PYZus{}split}\PY{p}{(}\PY{n}{TrainVecsN}\PY{p}{,}\PY{n}{test\PYZus{}size}\PY{o}{=}\PY{l+m+mf}{0.2}\PY{p}{)}
\end{Verbatim}


    \begin{itemize}
\tightlist
\item
  Berechnung der Mittelpunktsvektoren und Kovarianzmatrizen für P und N
\end{itemize}

    \begin{Verbatim}[commandchars=\\\{\}]
{\color{incolor}In [{\color{incolor}53}]:} \PY{n}{mu\PYZus{}p} \PY{o}{=} \PY{n}{MittelPunktVector}\PY{p}{(}\PY{n}{TrainVecsP}\PY{p}{)}
         \PY{n}{mu\PYZus{}n} \PY{o}{=} \PY{n}{MittelPunktVector}\PY{p}{(}\PY{n}{TrainVecsN}\PY{p}{)}
         
         \PY{n}{kov\PYZus{}p} \PY{o}{=} \PY{n}{getKovarianzMatrix}\PY{p}{(}\PY{n}{TrainVecsP}\PY{p}{,} \PY{n}{mu\PYZus{}p}\PY{p}{)}
         \PY{n}{kov\PYZus{}n} \PY{o}{=} \PY{n}{getKovarianzMatrix}\PY{p}{(}\PY{n}{TrainVecsN}\PY{p}{,} \PY{n}{mu\PYZus{}n}\PY{p}{)}
\end{Verbatim}


    \begin{itemize}
\tightlist
\item
  Berechnung von alpha mit den Kovarianzmatrizen \(\Sigma_P\),
  \(\Sigma_N\) und den Mittelpunktvektoren \(\vec \mu_P\),
  \(\vec \mu_N\) (c = 1, da \(\vec \alpha\) danach auf Länge 1 normiert
  wird und Skalierung damit egal )
\end{itemize}

\[ \vec \alpha = c (\Sigma_P + \Sigma_N)^{-1} (\vec \mu_P - \vec \mu_N) \]

\begin{itemize}
\tightlist
\item
  Berechnung der Varianzen \(\sigma^2\) für P und N
\end{itemize}

\[ \sigma^2 = \vec \alpha^T \Sigma  \vec \alpha \]

\begin{itemize}
\tightlist
\item
  Berechnung des projizierten Mittelwerts \(\mu_p\)
\end{itemize}

\[ \mu_p = \vec \mu \cdot \vec \alpha  \]

    \begin{Verbatim}[commandchars=\\\{\}]
{\color{incolor}In [{\color{incolor}54}]:} \PY{n}{alpha} \PY{o}{=} \PY{n}{pinv}\PY{p}{(}\PY{p}{(}\PY{n}{kov\PYZus{}p} \PY{o}{+} \PY{n}{kov\PYZus{}n}\PY{p}{)}\PY{p}{)} \PY{o}{@} \PY{n}{np}\PY{o}{.}\PY{n}{transpose}\PY{p}{(}\PY{n}{mu\PYZus{}p} \PY{o}{\PYZhy{}} \PY{n}{mu\PYZus{}n}\PY{p}{)}
         \PY{n}{alpha} \PY{o}{=} \PY{n}{alpha}\PY{o}{/}\PY{n}{norm}\PY{p}{(}\PY{n}{alpha}\PY{p}{)}
         \PY{n}{alpha} \PY{o}{=} \PY{n}{np}\PY{o}{.}\PY{n}{transpose}\PY{p}{(}\PY{n}{alpha}\PY{p}{)}
         
         \PY{n}{var\PYZus{}p} \PY{o}{=} \PY{n+nb}{float}\PY{p}{(}\PY{n}{np}\PY{o}{.}\PY{n}{transpose}\PY{p}{(}\PY{n}{alpha}\PY{p}{)} \PY{o}{@} \PY{n}{kov\PYZus{}p} \PY{o}{@} \PY{n}{alpha}\PY{p}{)}
         \PY{n}{var\PYZus{}n} \PY{o}{=} \PY{n+nb}{float}\PY{p}{(}\PY{n}{np}\PY{o}{.}\PY{n}{transpose}\PY{p}{(}\PY{n}{alpha}\PY{p}{)} \PY{o}{@} \PY{n}{kov\PYZus{}n} \PY{o}{@} \PY{n}{alpha}\PY{p}{)}
         
         \PY{n}{mu\PYZus{}p\PYZus{}projected} \PY{o}{=} \PY{n+nb}{float}\PY{p}{(}\PY{n}{np}\PY{o}{.}\PY{n}{dot}\PY{p}{(}\PY{n}{mu\PYZus{}p}\PY{p}{,}\PY{n}{alpha}\PY{p}{)}\PY{p}{)}
         \PY{n}{mu\PYZus{}n\PYZus{}projected} \PY{o}{=} \PY{n+nb}{float}\PY{p}{(}\PY{n}{np}\PY{o}{.}\PY{n}{dot}\PY{p}{(}\PY{n}{mu\PYZus{}n}\PY{p}{,}\PY{n}{alpha}\PY{p}{)}\PY{p}{)}
\end{Verbatim}


    \subsection{Plotten der beiden
Dichtefunktionen}\label{plotten-der-beiden-dichtefunktionen}

    \begin{Verbatim}[commandchars=\\\{\}]
{\color{incolor}In [{\color{incolor}55}]:} \PY{n+nb}{print}\PY{p}{(}\PY{l+s+s2}{\PYZdq{}}\PY{l+s+s2}{Klasse P}\PY{l+s+s2}{\PYZdq{}}\PY{p}{)}
         \PY{n+nb}{print}\PY{p}{(}\PY{l+s+s2}{\PYZdq{}}\PY{l+s+s2}{Mittelwert: }\PY{l+s+s2}{\PYZdq{}} \PY{o}{+} \PY{n+nb}{str}\PY{p}{(}\PY{n}{mu\PYZus{}p\PYZus{}projected}\PY{p}{)}\PY{p}{)}
         \PY{n+nb}{print}\PY{p}{(}\PY{l+s+s2}{\PYZdq{}}\PY{l+s+s2}{Varianz: }\PY{l+s+s2}{\PYZdq{}} \PY{o}{+} \PY{n+nb}{str}\PY{p}{(}\PY{n}{var\PYZus{}p}\PY{p}{)}\PY{p}{)}
         \PY{n+nb}{print}\PY{p}{(}\PY{l+s+s2}{\PYZdq{}}\PY{l+s+s2}{\PYZdq{}}\PY{p}{)}
         \PY{n+nb}{print}\PY{p}{(}\PY{l+s+s2}{\PYZdq{}}\PY{l+s+s2}{Klasse N}\PY{l+s+s2}{\PYZdq{}}\PY{p}{)}
         \PY{n+nb}{print}\PY{p}{(}\PY{l+s+s2}{\PYZdq{}}\PY{l+s+s2}{Mittelwert: }\PY{l+s+s2}{\PYZdq{}} \PY{o}{+} \PY{n+nb}{str}\PY{p}{(}\PY{n}{mu\PYZus{}n\PYZus{}projected}\PY{p}{)}\PY{p}{)}
         \PY{n+nb}{print}\PY{p}{(}\PY{l+s+s2}{\PYZdq{}}\PY{l+s+s2}{Varianz: }\PY{l+s+s2}{\PYZdq{}} \PY{o}{+} \PY{n+nb}{str}\PY{p}{(}\PY{n}{var\PYZus{}n}\PY{p}{)}\PY{p}{)}
         
         
         
         \PY{n}{x} \PY{o}{=} \PY{n}{np}\PY{o}{.}\PY{n}{linspace}\PY{p}{(}\PY{o}{\PYZhy{}}\PY{l+m+mf}{1.5}\PY{p}{,}\PY{l+m+mi}{2}\PY{p}{,}\PY{l+m+mi}{100}\PY{p}{)}
         \PY{n}{y} \PY{o}{=} \PY{n}{gauß}\PY{p}{(}\PY{n}{x}\PY{p}{,} \PY{n}{mu\PYZus{}p\PYZus{}projected}\PY{p}{,} \PY{n}{var\PYZus{}p}\PY{p}{)}
         \PY{n}{pylab}\PY{o}{.}\PY{n}{plot}\PY{p}{(}\PY{n}{x}\PY{p}{,}\PY{n}{y}\PY{p}{)}
         
         
         \PY{n}{x} \PY{o}{=} \PY{n}{np}\PY{o}{.}\PY{n}{linspace}\PY{p}{(}\PY{o}{\PYZhy{}}\PY{l+m+mf}{1.5}\PY{p}{,}\PY{l+m+mi}{2}\PY{p}{,}\PY{l+m+mi}{100}\PY{p}{)}
         \PY{n}{y} \PY{o}{=} \PY{n}{gauß}\PY{p}{(}\PY{n}{x}\PY{p}{,} \PY{n}{mu\PYZus{}n\PYZus{}projected}\PY{p}{,} \PY{n}{var\PYZus{}n}\PY{p}{)}
         \PY{n}{pylab}\PY{o}{.}\PY{n}{plot}\PY{p}{(}\PY{n}{x}\PY{p}{,}\PY{n}{y}\PY{p}{)}
         
         \PY{n}{pylab}\PY{o}{.}\PY{n}{show}\PY{p}{(}\PY{p}{)}
\end{Verbatim}


    \begin{Verbatim}[commandchars=\\\{\}]
Klasse P
Mittelwert: 0.8358656844402031
Varianz: 0.2076368497472233

Klasse N
Mittelwert: -0.13424976498791436
Varianz: 0.1609224943925902

    \end{Verbatim}

    \begin{center}
    \adjustimage{max size={0.9\linewidth}{0.9\paperheight}}{output_12_1.png}
    \end{center}
    { \hspace*{\fill} \\}
    
    \subsection{Projizierung und Klassifizierung der
Testdaten}\label{projizierung-und-klassifizierung-der-testdaten}

    \begin{Verbatim}[commandchars=\\\{\}]
{\color{incolor}In [{\color{incolor}56}]:} \PY{n}{success} \PY{o}{=} \PY{l+m+mi}{0}
         \PY{n}{tests} \PY{o}{=} \PY{n+nb}{len}\PY{p}{(}\PY{n}{TestVecsP}\PY{p}{)} \PY{o}{+} \PY{n+nb}{len}\PY{p}{(}\PY{n}{TestVecsN}\PY{p}{)}
         
         \PY{n}{konfmat} \PY{o}{=} \PY{p}{[}\PY{p}{[}\PY{l+m+mi}{0}\PY{p}{,}\PY{l+m+mi}{0}\PY{p}{]}\PY{p}{,}\PY{p}{[}\PY{l+m+mi}{0}\PY{p}{,}\PY{l+m+mi}{0}\PY{p}{]}\PY{p}{]}
         
         
         \PY{k}{for} \PY{n}{vector} \PY{o+ow}{in} \PY{n}{TestVecsP}\PY{p}{:}
             \PY{n}{proj\PYZus{}vector} \PY{o}{=} \PY{n+nb}{float}\PY{p}{(}\PY{n}{np}\PY{o}{.}\PY{n}{dot}\PY{p}{(}\PY{n}{vector}\PY{p}{,}\PY{n}{alpha}\PY{p}{)}\PY{p}{)}
             \PY{n}{prob\PYZus{}p} \PY{o}{=} \PY{n}{gauß}\PY{p}{(}\PY{n}{proj\PYZus{}vector}\PY{p}{,} \PY{n}{mu\PYZus{}p\PYZus{}projected}\PY{p}{,} \PY{n}{var\PYZus{}p}\PY{p}{)}
             \PY{n}{prob\PYZus{}n} \PY{o}{=} \PY{n}{gauß}\PY{p}{(}\PY{n}{proj\PYZus{}vector}\PY{p}{,} \PY{n}{mu\PYZus{}n\PYZus{}projected}\PY{p}{,} \PY{n}{var\PYZus{}n}\PY{p}{)}
             \PY{k}{if}\PY{p}{(}\PY{n}{prob\PYZus{}p} \PY{o}{\PYZgt{}} \PY{n}{prob\PYZus{}n}\PY{p}{)}\PY{p}{:}
                 \PY{n}{success} \PY{o}{+}\PY{o}{=} \PY{l+m+mi}{1}
                 \PY{n}{konfmat}\PY{p}{[}\PY{l+m+mi}{0}\PY{p}{]}\PY{p}{[}\PY{l+m+mi}{0}\PY{p}{]} \PY{o}{+}\PY{o}{=} \PY{l+m+mi}{1}
             \PY{k}{else}\PY{p}{:}
                 \PY{n}{konfmat}\PY{p}{[}\PY{l+m+mi}{0}\PY{p}{]}\PY{p}{[}\PY{l+m+mi}{1}\PY{p}{]} \PY{o}{+}\PY{o}{=} \PY{l+m+mi}{1}
         
         \PY{k}{for} \PY{n}{vector} \PY{o+ow}{in} \PY{n}{TestVecsN}\PY{p}{:}
             \PY{n}{proj\PYZus{}vector} \PY{o}{=} \PY{n+nb}{float}\PY{p}{(}\PY{n}{np}\PY{o}{.}\PY{n}{dot}\PY{p}{(}\PY{n}{vector}\PY{p}{,}\PY{n}{alpha}\PY{p}{)}\PY{p}{)}
             \PY{n}{prob\PYZus{}p} \PY{o}{=} \PY{n}{gauß}\PY{p}{(}\PY{n}{proj\PYZus{}vector}\PY{p}{,} \PY{n}{mu\PYZus{}p\PYZus{}projected}\PY{p}{,} \PY{n}{var\PYZus{}p}\PY{p}{)}
             \PY{n}{prob\PYZus{}n} \PY{o}{=} \PY{n}{gauß}\PY{p}{(}\PY{n}{proj\PYZus{}vector}\PY{p}{,} \PY{n}{mu\PYZus{}n\PYZus{}projected}\PY{p}{,} \PY{n}{var\PYZus{}n}\PY{p}{)}
             \PY{k}{if}\PY{p}{(}\PY{n}{prob\PYZus{}n} \PY{o}{\PYZgt{}} \PY{n}{prob\PYZus{}p}\PY{p}{)}\PY{p}{:}
                 \PY{n}{success} \PY{o}{+}\PY{o}{=} \PY{l+m+mi}{1}
                 \PY{n}{konfmat}\PY{p}{[}\PY{l+m+mi}{1}\PY{p}{]}\PY{p}{[}\PY{l+m+mi}{1}\PY{p}{]} \PY{o}{+}\PY{o}{=} \PY{l+m+mi}{1}
             \PY{k}{else}\PY{p}{:}
                 \PY{n}{konfmat}\PY{p}{[}\PY{l+m+mi}{1}\PY{p}{]}\PY{p}{[}\PY{l+m+mi}{0}\PY{p}{]} \PY{o}{+}\PY{o}{=} \PY{l+m+mi}{1}
         
         \PY{n+nb}{print}\PY{p}{(}\PY{l+s+s2}{\PYZdq{}}\PY{l+s+s2}{Tests: }\PY{l+s+s2}{\PYZdq{}} \PY{o}{+} \PY{n+nb}{str}\PY{p}{(}\PY{n}{tests}\PY{p}{)} \PY{o}{+} \PY{l+s+s2}{\PYZdq{}}\PY{l+s+s2}{ Erfolge: }\PY{l+s+s2}{\PYZdq{}} \PY{o}{+} \PY{n+nb}{str}\PY{p}{(}\PY{n}{success}\PY{p}{)} \PY{o}{+} \PY{l+s+s2}{\PYZdq{}}\PY{l+s+s2}{ Erfolgsrate: }\PY{l+s+s2}{\PYZdq{}} \PY{o}{+} \PY{n+nb}{str}\PY{p}{(}\PY{n}{success}\PY{o}{/}\PY{n}{tests}\PY{p}{)}\PY{p}{)}
         \PY{n+nb}{print}\PY{p}{(}\PY{p}{)}
         \PY{n+nb}{print}\PY{p}{(}\PY{l+s+s2}{\PYZdq{}}\PY{l+s+s2}{Konfusionsmatrix: }\PY{l+s+s2}{\PYZdq{}}\PY{p}{)}
         \PY{n+nb}{print}\PY{p}{(}\PY{n}{np}\PY{o}{.}\PY{n}{matrix}\PY{p}{(}\PY{n}{konfmat}\PY{p}{)}\PY{p}{)}
\end{Verbatim}


    \begin{Verbatim}[commandchars=\\\{\}]
Tests: 921 Erfolge: 840 Erfolgsrate: 0.9120521172638436

Konfusionsmatrix: 
[[325  38]
 [ 43 515]]

    \end{Verbatim}

    \subsubsection{Kann man anhand der gefundenen Projektion beurteilen
welche Merkmale am nützlichsten
sind?}\label{kann-man-anhand-der-gefundenen-projektion-beurteilen-welche-merkmale-am-nuxfctzlichsten-sind}

Das gezeigte Verfahren nimmt die Klassifizierung in einem 1-dim Raum
vor. Dieser Raum Entsteht durch projektion auf den Vektor
\(\vec \alpha\)

Beobachtung (1) : In einem 1-dim Raum ist der Fehler beim Klassifizieren
dann am kleinsten, wenn die Mittelpunkte \(\mu_P\) und \(\mu_N\) weit
entfernt sind, und die meisten Vektoren dicht an den Mittelpunkten
liegen, also:

\[
\begin{aligned}
    (\mu_P - \mu_N)^2 \rightarrow \text{ max } \\
    (\sigma_p + \sigma_n) \rightarrow \text{ min }
\end{aligned}
\]

Diese beiden Kriterien sind äquivalent zum Fischer-Kriterium:

\[
\begin{aligned}
&S(\vec \alpha) = \frac{(\mu_P - \mu_N)^2}{\sigma_P + \sigma_N} \rightarrow \text{ max } \\
\iff& \\
&\dots \\
\iff& \\
&\vec \alpha = c(\Sigma_P + \Sigma_N)^{-1} (\mu_P - \mu_N) \\
\end{aligned}
\]

Wir haben \(\vec \alpha\) bestimmt, so dass die Fischer-Diskriminante
\(S(\vec \alpha)\) maximal ist.

Also ist \(\vec \alpha\) entsprechend der Beoabachtung (1) auch die
Richtung, in der die Klassifikation den kleinsten Fehler aufweist.
Anders ausgedrückt: Die Merkmale deren Koordinatenachse in eine
möglichst ähnliche Richtung wie \(\vec \alpha\) oder \(-\vec \alpha\)
zeigen, sind die Merkmale, die am nützlichsten sind. Die Merkmale deren
Koordinatenachse eher senkrecht zur Richtung von \(\vec \alpha\) stehen
werden am wenigsten durch die Projektion widergespiegelt und sind somit
nicht so nützlich.


    % Add a bibliography block to the postdoc
    
    
    
    \end{document}
